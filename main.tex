%!TeX TS-program = Lualatex
%!TeX encoding = UTF-8 Unicode
%!TeX spellcheck = en
%!BIB TS-program = biber
% -*- coding: UTF-8; -*-
% vim: set fenc=utf-8
%%%%%%%%%%%%%%%%%%%%%%%%%%%%%%%%%%%%%%%%%%%%%%%%%%%%%%%%%%%%%%%%%%%%%
\documentclass[11pt]{article}
\usepackage[utf8]{inputenc}
\usepackage{times}
\usepackage{wrapfig}
\usepackage{graphicx}
\usepackage{floatrow}
\usepackage{multicol}
\usepackage{amsmath}


\oddsidemargin -0.5in		% margin, in addition to 1" standard
\textwidth 7.5in		% 8.5" - 2*(1+\oddsidemargin)

\topmargin -1in		% in addition to 1.5" standard margin
\textheight 9.75in 		% 11 - ( 1.5 + \topmargin + <bottom-margin> )

\columnsep 0.25in

\parindent 0pt
\parskip 12pt

\flushbottom \sloppy
\pagestyle{empty} % No page numbers

\usepackage{siunitx}%The siunitx package provides a  set  of  tools  for  authors  to  typeset  quantities  in  aconsistent  way.

\newcommand{\ms}{\si{\milli\second}}%

%%%%%%%%%%%%%%%%%%%%%%%%%%%%%%%%%%%%%%%%%%%%%%%%%%%%%%%%%%%%%%%%%%%%%
\usepackage[
%style=chem-acs,
style=numeric,						% numeric style for reference list
citestyle=numeric-comp,
%style=alphabetic-verb,
giveninits=false,
maxbibnames=1,
%firstinits=true,
%style=apa,
%maxcitenames=1,
%maxnames=3,
%minnames=1,
%maxbibnames=99,
dateabbrev=true,
giveninits=true,
%uniquename=init,
url=false,
doi=false,
isbn=false,
eprint=false,
texencoding=utf8,
bibencoding=utf8,
autocite=superscript,
backend=biber,
%sorting=none,
sorting=none,
sortcites=false,
%articletitle=false
]{biblatex}%

\bibliography{ref.bib}
%%%%%%%%%%%%%%%%%%%%%%%%%%%%%%%%%%%%%%%%%%%%%%%%%%%%%%%%%%%%%%%%%%%%%%
\newcommand{\mycaption}[1]{\caption*{#1}}

\usepackage{titlesec}
% \titlespacing*{<command>}{<left>}{<before-sep>}{<after-sep>}
\titlespacing*{\section}
{0pt}{1.5ex}{0.8ex}
\titlespacing*{\subsection}
{0pt}{0.9ex}{0.4ex}
\titlespacing*{\subsubsection}
{0pt}{0.5ex}{0.3ex}
\titlespacing*{\paragraph}{%
  0pt}{%              left margin
  0.0\baselineskip}{% space before (vertical)
  1em}%               space after (horizontal)

\usepackage{setspace}

\begin{document}

%%%-----------------------------------------------------------------
{\Large\bf 
Automatic detection of spiking motifs in multi-unit raster plots
}

%{\bf
%Anonymous submission for double-blind review. \\
%}
\hrule
%%SUMMARY
%%%-----------------------------------------------------------------
\textbf{Summary} %}
Recently, there has been growing interest in the detection of repeating spike patterns in multi-unit raster plots. In this study, we introduce an extension of such detection models by providing a generative model for raster plot synthesis. An optimal detection procedure is derived from this model. This takes the form of a logistic regression coupled with a temporal convolution. We evaluate the ability of this model to detect spike patterns in synthetic data. Since this model is differentiable, we derive an unsupervised learning method in the form of gradient descent on the loss function of an auto-encoder model for the raster using the spike patterns. This unsupervised learning method is able to recover the synthetically generated spike patterns, and we plan to apply it to neurobiological data as well.
\vspace{.5cm}
\hrule
%---------------------------
\textbf{Additional Details.}%
%
\begin{figure}[h!]%wrapfigure}{}{\textwidth}
  % \subfloat[M]{%ulti-unit raster]{
    \includegraphics[width=.245\linewidth]{figure_1a_k.pdf}
  % }%
  % %\hfill
  % \subfloat[M]{%Spiking motifs]{
    \includegraphics[width=.245\linewidth]{figure_1b.pdf}
  % }%
  % \subfloat[M]{%Raster of motifs]{
  \includegraphics[width=.245\linewidth]{figure_1c.pdf}
  % }%
  % %\hfill
  % \subfloat[M]{%Annotated raster]{
    \includegraphics[width=.245\linewidth]{figure_1a.pdf}
  % }
  %\vspace{-55pt}
{
\caption{A multi-unit raster plot is the superposition of different spiking motifs. We show these different motifs, each identified on top by a different color, as probabilities of activation (red) or deactivation (blue), on the space of $10$ neurons from the multi-unit recording, and $71$ different possible delays. This superposition is defined by a generative model for the activation in time of the different motifs which is then used to draw a raster plot on the the multi-unit address space. Inverting this model allow to define an inference model for their efficient detection and to annotate the original raster plot with the identified spiking motif.
}
\label{fig:1}
}
\vspace{-5pt}
\end{figure}%wrapfigure}
%---------------------------

%
% * Limit : not online - in the future it is a model of a neuron
%
Recent technological advances has allowed to record simultaneously a large number of neurons, leading the way for the development of efficient methods for detecting structured patterns in raster plots~\parencite{russo_cell_2017, stella_3d-spade_2019}. Here, we will try to extend these methods by explicitly representing spiking motifs as a temporal sequence of activation of the different units. Thus, we can parameterize each motif by the set of tuples defining the weight and delay of each element of the motif. By discretization of time (with here an arbitrary unitary time unit of $1~\ms$), we can also define the motifs as a matrix giving the weight corresponding to the different delays $d \in [0, D]$ (where $D$ is the maximum delay) on different addresses $a \in [1, N]$ defining the list of the $N$ neurons from the multi-unit recording. We will denote as $W(a, d)$ these weights. When a given motif is activated, it will generate a discharge pattern that corresponds to this specific set of delays and addresses. Assuming that we know there exists $M$ such motifs, we will define as $b \in [1, M]$ the address of a PG and as $W_b$ the corresponding weight matrices. This allows then to derive a generative model for raster plots (see Figure~\ref{fig:1}).

Indeed, the probability of firing of a neuron $a$ at a given time $t$ can be understood as a Bernoulli trial whose (only) parameter is a bias $p(t, a) \in [0, 1]$. Assuming that the presence of spiking motifs is conditionally independent, this bias can be written as the combination of these factors, whose values are given by the corresponding weights. spiking motifs may be activated independently at random times and  we write that $B(b, t)=1$ if $b$ is activated at $t$ (and else $B(b, t)=0$). We can thus write the probability bias as %the joint probability given these factors as 
$%$
p(t, a) = \sigma\big(W_0 + \sum_{b, t} B(b, t) \cdot W_b(a, t-d) \big)  
$, %$
where $\sigma$ is the sigmoid function. We will further assume that the weights are balanced (their mean is zero) and that $W_0$ is a bias such that $p_0=\sigma(W_0)$ is the average background firing rate. Conveniently, one can write this summation as a one-dimensional temporal convolution operator such that we may simply write $p = \sigma(W_0 + B \ast W )$ where  $p\in [ 0, 1]^{N\times T}$ and $B\in \{0, 1\}^{M\times T}$ is the raster plot corresponding to the temporal activation of the spiking motifs. Finally, we obtain the raster plot $A\in \{0, 1\}^{N\times T}$ by drawing spikes using independent Bernoulli trials $A \sim \mathcal{B}(p)$. Note that, depending on the shape of the kernels, the generative model can model a Poisson process, generate rhythmic activity or more generally propagating waves. This formulation thus defines a simple generative model for raster plots as a combination of independent spiking motifs. 

%---------------------------
\begin{wrapfigure}{r}{.275\textwidth}
\vspace{-15pt}
% ❯ git add figure_N_PG_time.pdf figure_N_PGs.pdf figure_N_pre.pdf
%\subfloat[]{
\includegraphics[width=\linewidth]{figure_N_PGs.pdf}
%}%
%\subfloat[]{\includegraphics[width=.49\linewidth]{Figures/fig_methods_nat.pdf}}%
\vspace{-25pt}
{
\caption{Accuracy of PG detection as a function of the number $M$ of kernels.
}
\label{fig:2}
}
\vspace{-10pt}
\end{wrapfigure}
%---------------------------
This generative model allows to define an inference model for guessing sources $B$ when observing a raster plot $A$. This assumes that we know the spiking motifs as defined by the $W_b$ matrices. The underlying metric is the binary cross-entropy, as used in the logistic regression model. In particular, if we consider PG kernels with similar decreasing exponential time profile, we can prove that this is similar to finding the tuning function of L-NL neurons, as used in the method of~\parencite{berens_fast_2012}. In our specific case, the difference is that the regression is performed in both dendritic and delay space by extending the summation using a temporal convolution operator. Using this forward model, it possible to estimate the logit (inverse of a sigmoid) $\hat{B}(b, t)$ for the presence of a PG of address $b$ and at time $t$ by using the transpose convolution operator. It thus comes that when observing $A$, then one may infer $\hat{B} = A \ast W^T$ and select the most activated items. To quantify the efficiency of this operation, we generated $M=55$ synthetic spiking motifs as random independent kernels over $128$ presynaptic inputs and $D=127$ possible delays. We drew random independent instances of $B$ with a length of $T=1000$ time steps and with on average $2.0$ occurrences each. This allowed us to generate raster plots which we use to infer $\hat{B}$. We compute the accuracy as the rate of true positive detections and observe on average $\approx 98\%$ correct spiking motifs both for the address and %(exact) 
timing.

%---------------------------
\begin{wrapfigure}{r}{.275\textwidth}
\vspace{-10pt}
% ❯ git add figure_N_PG_time.pdf figure_N_PGs.pdf figure_N_pre.pdf
%\subfloat[]{
\includegraphics[width=\linewidth]{figure_N_PG_time.pdf}
%}%
%\subfloat[]{\includegraphics[width=.49\linewidth]{Figures/fig_methods_nat.pdf}}%
\vspace{-25pt}
{
\caption{Accuracy of PG detection as a function of the temporal depth $D$ of kernels.
}
\label{fig:3}
}
\vspace{-15pt}
\end{wrapfigure}
%---------------------------
We further extended this result by showing how the accuracy would evolve as a function of the number of simultaneous spiking motifs, while keeping the same frequency of occurrence. We show in Figure~\ref{fig:2} that the accuracy of finding the right PG is still above $80\%$ accuracy with more than $1364$ spiking motifs. Moreover, we show in Figure~\ref{fig:3} that (with $M=55$ spiking motifs fixed) the accuracy increases notably as the temporal depth $D$ of the PG kernel increased, demonstrating quantitatively the potential of hetero-synaptic delays. These results were obtained while assuming that we know $W$. However, this is in general not the case, for instance when observing the raster plot of a population of neurons. Inspired by the k-means algorithm, it is possible to devise a self-supervised learning algorithm. Our preliminary results show that it is possible to retrieve spiking motifs embedded in the data, yet that further analysis is necessary to improve the convergence of the algorithm. In particular, it seems promising to use a sparseness constraint in the inference mechanism such as to remove spurious correlations in the inference.


% \section{Theoretical framework for polychronous group detection}


%\vspace{-55pt}
%\subsubsection*{References}
% \textbf{References.} %}
%{
%\small
%\footnotesize
%\begingroup
%\setstretch{0.75}
%\setlength\bibitemsep{1pt}
%\begingroup
%\setstretch{0.75}
%\setlength\bibitemsep{1pt}
%\printbibliography[heading=none]
%\printbibliography
%\endgroup
%\endgroup
%}
% \cite{Izhikevich06}~\href{http://izhikevich.org/publications/spnet.htm}{Izhikevich, \emph{Neural Computation}, 2006.}
% \cite{berens_fast_2012}~\href{https://www.jneurosci.org/content/32/31/10618}{Berens et al., \emph{Journal of Neuroscience}, 2012.}
% \end{document}

%
\printbibliography
\end{document}
